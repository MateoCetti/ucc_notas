\title{Resumen de algebra (segundo semestre)}

\author{Mateo P. Cetti}

\documentclass[10pt]{article}

\usepackage{amsmath}
\usepackage{amsfonts}

\begin{document} 
\maketitle

\section{Espacios vectoriales}

cualquier conjunto que posea operaciones de \textbf{suma} y 
\textbf{producto por un escalar}, cumpliendo todas las siguientes \textbf{propiedades:}\\
\linebreak
\paragraph{Suma: }

\begin{enumerate}
    \item \textbf{Asociativa} $(u+w) + v = u (w+v) = (u+v) + w$
    \item \textbf{Conmutativa} $u + v = v + u $
    \item \textbf{Elemento neutro} tal que $u + 0 = u$ 
    \item Para cada vector $u$ existe un elemento opuesto $(-u)$ tal que $u+(-u)=0$.
    
\end{enumerate}

\paragraph{Multiplicacion: }

\begin{enumerate}
    \item \textbf{Asociativa} $(k*k')*u=k*(k'*u)=(k*u)*k'  k,k' e \mathbb{K}$
    \item \textbf{Distributiva}
    \begin{itemize}
        \item Respecto a la suma de vectores  $k*(u+v) = k*u+k*v$
        \item Respecto a la suma de escalares  $(k_1+k_2) *u = k_1*u + k_2 *u$   
    
    \end{itemize}
    \item \textbf{Elemento neutro} $k$=1, tal que $1*u = u$
    
\end{enumerate}

Sea $\mathbb{K}$ un cuerpo, y $V$ un conjunto de dos operaciones definimos:

\paragraph{Operacion interna}llamada adición, que asigna a cada par 
$u,v$ de elementos de V, un elemento de V denotado $u+v$

\paragraph{Operacion externa}llamada multiplicación por un escalar que 
asigna a cada par formado por un elemento $ke\mathbb{K}$ y un elemento $veV$, un elemento de V denotado por $kv$

\paragraph{Subespacios vectoriales}

Un subconjunto no vacio W de un espacio vectorial V se denomina subespacio vectorial
de V, si W es en si mismo un espacio vectorial bajo las operaciones de suma y Multiplicacion
por escalar definidas en V. Ademas debe cumplir:

\begin{itemize}
    \item La suma de 2 vectores de W pertenecen a W
    $u,v \mathbb{e} W => u + v \mathbb{e} W$
    \item El producto de un vector de W por un escalar cualquiera K pertenece a W
    $u \mathbb{e} W, k \mathbb{e} \mathbb{K} => k.u  \mathbb{e} W$
    \item El vector nulo pertenece a W 
\end{itemize}

\section{Clase 2}

\paragraph{Combinacion lineal}

Sea $V$ un espacio vectorial  sobre un cuerpo $\mathbb{K}$, $v_1, v_2, v_n$ son vectores de $V$
$k_1, k_2,..., k_n$ son escalares de $\mathbb{K}$.\\
\linebreak
Se dice que un vector $v \in V$ es combinacion lineal de los vectores $v_1, v_2, v_n$ segun los escalares $k_1, k_2, k_n$ si y solo si: 
\begin{equation*}
	v = k_1 v_1+ k_2 v_2+k_n+v_n 
\end{equation*}

\paragraph{Generador de un subespacio vectorial} Sea $V$ un espacio vectorial sobre un cuerpo $\mathbb{K}$,
$v_1, v_2, v_n$ son vectores de $V$, $W$ es el conjunto de \textbf{todas} las combinaciones lineales de $v_1, v_2, v_n$.
Llamaremos a W \textbf{subespacio generado} por $v_1,v_2, v_n$ y lo indicaremos como:
\begin{center}
	$W = <v_1, v_2, v_n>$
\end{center}

y se lee "W es el subespacio generado por $v_1, v_2, v_n$"\\
\linebreak
El subespacio generado por los vectores $v_1, v_2, v_n$, puede ser el mismo $V$, en este caso diremos que el conjunto $S = {v_1, v_2,..., v_n}$ genera el espacio vectorial V o bien, S es un generador de $V$.\\
\linebreak
Sea $V$ un espacio vectorial sobre un cuerpo $\mathbb{K}$, el conjunto $S = {v_1, v_2, v_n}$ es un \textbf{generador} de V si y solo si $v \in V \rightarrow v = k_1v_1 + k_2v_2 + k_nv_n$

\begin{itemize}
	\item Un generador es un conjunto de vectores tales que todo vector del espacio vectorial se puede expresar como una combinacion lineal de ellos
	\item El subespacio generado es el conjunto formado por todas las 
\end{itemize}

\paragraph{Suma de conjuntos}
A y B son 2 conjuntos\\
$A+B = {x=a+b / a \in A y b \in B}$\\

La suma de 2 conjuntos es un nuevo conjunto formado por elementos donde se suma 1 elemento del primer conjunto con 1 elemento del segundo conjunto.

\paragraph{Suma de subespacios vectoriales}
sea V un espacio vectorial sobre un cuerpo $\mathbb{K}$ si $"w_1$ y $w_2$ son subespacios del espacio vectorial V, entonces $W_1 + W_2$ es un subespacio.

\paragraph{Interseccion de subespacios vectoriales}
sea V un espacio vectorial sobre un cuerpo $\mathbb{K}$ si $"w_1$ y $w_2$ son subespacios del espacio vectorial V, entonces $W_1 \cap W_2$ es un subespacio.

\section{Clase 3 (Teoremas "utiles")}

\paragraph{Teorema 1 (Combinacion lineal)}
Sea $V$ un espacio vectorial sobre un cuerpo $\mathbb{K}$ y sean $v_1, v_2, v_n$ vectores de $V$. Si $W$ es el conjunto de todas las combinaciones lineales de $v_1, v_2, v_n$ se verifica:

\begin{itemize}
	\item $W$ es un subespacio de $V$
	\item $v_1, v_2, v_n$ son elementos de $W$
	\item si $W'$ es cualquier subespacio que contiene a $v_1, v_2, v_n$ entonces $W \subset W'$
\end{itemize}

\paragraph{Teorema 2 (generador de un subespacio vectorial)}

Sea $V$ un espacio vectorial sobre un cuerpo $\mathbb{K}$.\\
$W = <v_1, v_2, v_n>$.\\
Si $v_1$ es combinación lineal de $ v_2, v_n$, entonces: $<v_1, v_2, v_n> = < v_2, v_3, v_n>$

\begin{center}
Como consecuencia de este teorema, si entre vectores de un \textbf{conjunto generador}, uno de ellos es \textbf{combinacion lienal} de los demas, este vector puede ser \textbf{eliminado} y los restantes siguen generando el mismo subespacio.
\end{center}

\paragraph{Dependencia e independencia lineal}

\paragraph{•} sea V un espacio vectorial sobre un cuerpo $\mathbb{K}$ y $v_1, v_2, v_n$ vectores de V el conjunto:
\begin{itemize}
	\item \textbf{a} ${v_1, v_2, v_n}$ es \textbf{linearmente dependiente} si y solo si el vector nulo se expresa \textbf{de mas de una forma} como combinacion lineal, es decir, existen escalares $k_1, k_2, k_n$ donde \textbf{NO TODOS son nulos}, tales que: $k_1 v_1 + k_2v_2 + ... + k_nv_n = 0$
	\item \textbf{b} ${v_1, v_2, v_n}$ es \textbf{linearmente independiente} si y solo si el vector nulo se expresa \textbf{de una unica forma} como combinacion lineal, es decir, existen escalares $k_1, k_2, k_n = 0$ tal que: $0 v_1 + 0v_2 + ... + 0v_n = 0$
\end{itemize}

\paragraph{Teoremas de caracterizacion}

\textbf{Teorema: }sea $V$ un espacio vectorial sobre un cuerpo $\mathbb{K}$ y sean $v_1, v_2, v_n$ vectores de $V$ con $n \geq 2$. El conjunto ${v_1,v_2,v_n}$ es \textbf{Linearmente dependiente} si y solo si alguno de ellos es combinacion lineal de los restantes.\\
\linebreak
\textbf{Teorema} sea $V$ un espacio vectorial sobre un cuerpo $\mathbb{K}$ y sean $v_1, v_2, v_n$ vectores de $V$ con $n \geq 2$. El conjunto ${v_1,v_2,v_n}$ es \textbf{Linearmente independiente} si y solo si todo $v \in <v_1, v_2, v_n>$ puede ser expresado como $v = k_1v_1 + k_2v_2+k_nv_n$ de forma \textbf{unica}.\\

\section{Clase 4}

\paragraph{Generadores y dependencia o independencia lineal
}
sea $V$ un espacio vectorial sobre un cuerpo $\mathbb{K}$ generado por los vectores $v_1$, $v_2$,...,$v_m$, 
$w_1,w_2,...w_n$ son elementos \textbf{arbitrarios} de $V$. Si $n > m$ entonces $w_1, w_2, 2_n$ son \textbf{linearmente dependientes}\\
\linebreak
Entre los \textbf{conjuntos generadores} de un espacio vectorial juegan un papel funamental aquellos que son \textbf{linearmente independientes}, por este motivo, los distinguiremos con un nombre especial: \textbf{Base}.\\
\linebreak
Sea $V$ un espacio vectorial sobre $\mathbb{K}$ el conjunto de vectores ${v_1, v_2, v_n}$ es una \textbf{base} de $V$ si y solo si:
\begin{itemize}
	\item $v_1, v_2, v_n$ son linearmente \textbf{independientes}
	\item $V = {v_1, v_2, v_n}$
\end{itemize}

\paragraph{Base canonica}
Todo conjunto de $n$ vectores linearmente independientes en $\mathbb{R}^n$ es una base en $\mathbb{R}^n$

\paragraph{Base ordenada}
Llamaremos base ordenada de $V$ a una sucesion finita de vectores $(v_1, v_2, v_n)$ linearmente independientes que generan a $V$ tales que ${v_1,v_2,v_n}$ es una base
\paragraph{Teorema}
Sea $V$ un espacio vectorial sobre un cuerpo $\mathbb{K}$. Si $B = {v_1, v_2, v_n}$ y $B' = {u_1,u_2,u_m}$ son bases de $V$ entonces $n=m$\\
Dos bases cualesquiera de un mismo espacio vectorial $V$ tienen el \textbf{mismo} numero de vectores.
\paragraph{Dimension}
El numero $n$ (entero no negativo) se llama \textbf{dimension} del espacio vectorial $V$. Asi $Dim(V)=n$ siendo $n$ el numero de vectores que forman sus bases\\
\linebreak
Sea $V$ un espacio vectorial sobre un cuerpo $\mathbb{K}$ de dimension finita n. Entonces se verifica:
\begin{itemize}
	\item Cualquier subconjunto de $V$ con mas  de $n$ elementos es linearmente \textbf{dependiente}
	\item Ningun subconjunto de $V$ con menos de $n$ elementos genera a $V$
	\item Todo subconjunto de $n$ vectores linearmente independientes es una \textbf{base} de $V$
	\item Todo generador de $V$, con $n$ vectores es una \textbf{base}
\end{itemize}

\paragraph{Existencia de bases}

\paragraph{Teorema}
en un espacio vectorial de dimensión finita todo subconjunto no vacío linealmente independiente es parte de una base

\paragraph{Teorema}
Todo generador finito de $V$ incluye una base\\
\linebreak
En todos los casos la dimension del subespacio de soluciones  es igual al numero de incognitas no principales
\paragraph{Teorema}
Sea $V$ un espacio vectorial sobre un cuerpo $\mathbb{K}$. Si $W_1$ y $W_2$ son subespacios de dimension finita, entonces ($W_1+W_2$) es un subespacio de dimension finita y se verifica: 
\begin{equation*}
	dim(W_1+W_2) = dim(W_1) + dim(W_2) - dim(W_1 \cap W_2)
\end{equation*}

\section{Clase 5}

\paragraph{suma directa}

Sean $W_1$ y $W_2$ subespacios vectoriales de $V$, La suma $W_1 + W_2$ es \textbf{directa} si y solo si todo vector de $W_1 + W_2$ se expresa en una \textbf{unica} forma como suma de un elemento de $W_1$ y otro de $W_2$ y (siendo $u_1, v_1 \in W_1$ y $u_2, v_2 \in W_2$) $u_1 = v_1$ y $u_2 = v_2$. La suma directa se denota como $W_1 \oplus W_2$

\paragraph{Teorema de caracterizacion de la suma directa}

Sean $W_1$ y $W_2$ subespacios vectoriales de $V$. La suma $W_1 + W_2$ es \textbf{directa} si y solo si $W_1 \cap W_2 = {0}$ 

\paragraph{Coordenadas de un vector respecto a una base}

Sea $B = (v_1, v_2, v_n)$ una base \textbf{ordenada} del espacio vectorial $V$ de dimension finita, entonces rodo vector de $V$ se expresa de forma unica como combinacion lineal de los vectores de la base, esto es, existen escalares unicos $k_1, k_2, k_n \in  \mathbb{K}$ tales que $v = k_1 v_1 + k_2 v_2 + k_n v_n$.\\
\linebreak
Entonces, los escalares $k_1, k_2, k_n$ son las \textbf{coordenadas del vector $V$ respecto de la base $B$}\\
\linebreak
Con estos escalares se puede armar una n-upla que recibe el nombre de: n-upla de coordenadas del vector $V$ respecto de la base $B$ y se denota:
\begin{itemize}
	\item $(v)_B = (k_1, k_2, k_n)$
	\item $[v]_B = \left[ k_1\\ k_2\\k_n  \right]$ (es una matriz de una sola columna :p)
\end{itemize}

\paragraph{Propiedades}

Sean $u,v \in V$ y $B$ una base ordenada de B entonces:

\begin{itemize}
	\item $(u+v)_B = (u)_B+(v)_B$
	\item $(ku)_B = k(u)_B$
\end{itemize}

\paragraph{Cambio de base}

La representación de vectores de un espacio vectorial $V$ de dimensión finita para sus vectores de coordenadas \textbf{depende de la base elegida}.
\begin{equation*}
	[v]_b = P \cdot [v]_{b'}
\end{equation*}
La matriz $P$ es cuadrada de orden $n*n$, es decir, $dim(P) = nxn$ siendo $n$ la dimension de $V$. P es \textbf{inversible}, y $[v]_{b'} = p^{-1} \cdot [v]_b$


\section{Aplicaciones lineales}

Sean $V \textbf{ y } W$ dos espacios vectoriales, una aplicacion lineal $F$ de $V$ en $W$ es una funcion que asigna a cada vector $v$ un vector unico $F(v) \in W$. Una funcion es una aplicacion lineal si y solo si:

\begin{itemize}
	\item $F(u+v) = F(u) + F(v)$
	\item $F(kv) = kF(v)$
\end{itemize}

\paragraph{Propiedades}

\begin{enumerate}
	\item $F(0_V) = F(0_W)$
	\item $F(-v) = -F(v)$
	\item $F(\alpha_1 v_1 + \alpha_2 v_2 + \alpha_3 v_3 + \alpha_n v_n) = \alpha_1 F(v_1) + \alpha_2 F(v_2) + \alpha_3 F(v_3) + \alpha_n F(v_n)$
\end{enumerate}

\paragraph{Teorema de la existencia y unicidad de la aplicacion lineal}

Sean $V$ y $W$ espacios vectoriales sobre $\mathbb{K}$, si $B = (v_1, v_2, v_n)$ es una base ordenada de $V$ y $w_1, w_2, w_n$ son vectores arbitrarios de $W$, entonces existe una \textbf{unica} aplicacion $F: V \rightarrow W$ tal que $F$ es \textbf{lineal} y $F(v_i) = w_i$
\paragraph{Observacion} Si se conoce el efecto de una transformación lineal sobre los vectores de la base, se puede conocer el efecto sobre cualquier otro vector.

\section{Nucleo e imagen de una aplicacion lineal}

\paragraph{Imagen de una aplicacion lineal}

Sea $F: V \rightarrow W$ una aplicacion lineal, el conjunto de todos los vectores de $W$ que son imagenes bajo $F$ de algun vector $v \in V$ se conoce como \textbf{imagen} de $F$ y se indica $I_F$
\paragraph{Teorema} Sea $F: V \rightarrow W$ una aplicacion lineal entonces:
\begin{itemize}
	\item $I_F$ es un \textbf{subespacio} de $W$
	\item si $v_1, v_2, v_n$ generan a $V$, entonces $F(v_1), F(v_2), F(v_n)$ generan a $I_F$
	\item Si $dim(V)=n$, entonces $dim(I_F) \leq n$
\end{itemize}

\paragraph{Nucleo de una aplicacion lineal} Sea $F: V \rightarrow W$ una aplicacion lineal, el \textbf{nucleo} de la aplicacion lineal $F$ esta formado por todos los vectores de $V$ cuya imagen es el $\overline{0_W}$ y se indica $N_F$.\\
$N_F$ \textbf{nunca es vacio} (nulo).

\paragraph{Teorema} Sea $F:V\rightarrow W$ una aplicacion lineal y $N_F = \left\lbrace v \in V / F(v) = \overline{0_W} \right\rbrace $ entonces:
\begin{itemize}
	\item $N_F$ es un \textbf{subespacio} de $V$
	\item Si $dim(v) = n$ entonces $dim n_F \leq n$
\end{itemize}

\paragraph{Teorema} Sea $f:V \rightarrow W$ una aplicacion lineal, si $V$ es de dimension finita, entonces $dim(V) = dim(N_F) + dim(I_F)$

\section{Tipos de aplicaciones lineales}

\paragraph{Aplicacion lineal inyectiva}

Sea $F:V\rightarrow W$ una aplicacion lineal, esta es \textbf{inyectiva} si y solo si a todo vector $w$ en la imagen de $F$ le corresponde \textbf{exactamente un} vector de $V$\\
\linebreak
\textbf{Teorema} Sea $F:V\rightarrow W$ una aplicacion lineal, $F$ es inyectiva si y solo si $N_F = {\overline{0_V}}$

\paragraph{Aplicacion lineal suryectiva}

Sea $F:V\rightarrow W$ una aplicacion lineal, $F$ es \textbf{suryectiva} si $I_F = W$ (todos los vectores de $W$ son imagen de \textbf{algun} vector de $V$) 

\paragraph{Aplicacion lineal biyectiva}

Sea $F:V\rightarrow W$ una aplicacion lineal, $F$ es \textbf{biyectiva} si y solo si es \textbf{inyectiva} y \textbf{suryectiva} a  la vez.\\
\linebreak
\textbf{teorema} Sea $F:V\rightarrow W$ una aplicacion lineal, si $dim(V) = dim(W) = n$ entonces $F$ es \textbf{biyectiva.} 

\paragraph{Aplicacion lineal inversible}

Sea $F:V\rightarrow W$ una aplicacion lineal, si $F$ es biyectiva, entonces:

\begin{itemize}
	\item Esta definida la aplicacion inversa $F^{-1}: W \rightarrow V$
	\item $F^{-1}$ es biyectiva
	\item La imagen inversa de $W$ (o sea de $V$) es $W$
\end{itemize}

\paragraph{Operaciones con aplicaciones lineales}.\\

$F+T: U \rightarrow W$\\
$(F+T)(u) = F(u) + T(u)$\\
\linebreak
$k\cdot F: V \rightarrow W$\\
$(KF)(u) = kF(u)$

\section{Composicion, vectorial, operadores (?)}

\paragraph{Composicion de aplicaciones lineales} $F: U \rightarrow V$ y $G: V \rightarrow W$ son apl lineales arbitrarias, se define la compuesta de $G$ con $F$:
\begin{equation*}
	G \circ F: U \rightarrow W
\end{equation*}
\begin{equation*}
	(G \circ F) (u) = G(F(u))
\end{equation*}
\textbf{Teorema} Si $F$ y $G$ son \textbf{lineales}, entonces $(G \circ F)$ tambien es lineal

\paragraph{Propiedades}
\begin{itemize}
	\item \textbf{Asociativa} $K \circ (G \circ F) = (H \circ G) \circ F$
	\item \textbf{Distributiva} $G \circ (F_1 + F_2) = G\circ F_1 + G \circ F_2$
\end{itemize}

\paragraph{El vectorial} Sean $V$ y $W$ espacios vectoriales sobre $K$ definimos $L(V,W)$ al conjunto de todas las aplicaciones lineales de $V$ en $W$.\\
Se pueden sumar y multiplicar vectoriales por un escalar

\paragraph{Operadores lineales L(V)} Sea $V$ un espacio vectorial sobre $K$ llamamos $L(V)$ al conjunto de todas las aplicaciones lineales $F:V \rightarrow V$

\end{document}