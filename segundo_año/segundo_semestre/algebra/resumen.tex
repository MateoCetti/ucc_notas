\title{Resumen de algebra (segundo semestre)}

\author{Mateo P. Cetti}

\documentclass[10pt]{article}

\usepackage{amsmath}
\usepackage{amsfonts}

\begin{document} 
\maketitle

\section{Espacios vectoriales}

cualquier conjunto que posea operaciones de \textbf{suma} y 
\textbf{producto por un escalar}, cumpliendo todas las siguientes \textbf{propiedades:}\\
\linebreak
\paragraph{Suma: }

\begin{enumerate}
    \item \textbf{Asociativa} $(u+w) + v = u (w+v) = (u+v) + w$
    \item \textbf{Conmutativa} $u + v = v + u $
    \item \textbf{Elemento neutro} tal que $u + 0 = u$ 
    \item Para cada vector $u$ existe un elemento opuesto $(-u)$ tal que $u+(-u)=0$.
    
\end{enumerate}

\paragraph{Multiplicacion: }

\begin{enumerate}
    \item \textbf{Asociativa} $(k*k')*u=k*(k'*u)=(k*u)*k'  k,k' e \mathbb{K}$
    \item \textbf{Distributiva}
    \begin{itemize}
        \item Respecto a la suma de vectores  $k*(u+v) = k*u+k*v$
        \item Respecto a la suma de escalares  $(k_1+k_2) *u = k_1*u + k_2 *u$   
    
    \end{itemize}
    \item \textbf{Elemento neutro} $k$=1, tal que $1*u = u$
    
\end{enumerate}

Sea $\mathbb{K}$ un cuerpo, y $V$ un conjunto de dos operaciones definimos:

\paragraph{Operacion interna}llamada adición, que asigna a cada par 
$u,v$ de elementos de V, un elemento de V denotado $u+v$

\paragraph{Operacion externa}llamada multiplicación por un escalar que 
asigna a cada par formado por un elemento $ke\mathbb{K}$ y un elemento $veV$, un elemento de V denotado por $kv$

\paragraph{Subespacios vectoriales}

Un subconjunto no vacio W de un espacio vectorial V se denomina subespacio vectorial
de V, si W es en si mismo un espacio vectorial bajo las operaciones de suma y Multiplicacion
por escalar definidas en V. Ademas debe cumplir:

\begin{itemize}
    \item La suma de 2 vectores de W pertenecen a W
    $u,v \mathbb{e} W => u + v \mathbb{e} W$
    \item El producto de un vector de W por un escalar cualquiera K pertenece a W
    $u \mathbb{e} W, k \mathbb{e} \mathbb{K} => k.u  \mathbb{e} W$
    \item El vector nulo pertenece a W 
\end{itemize}

\section{Clase 2}

\paragraph{Combinacion lineal}

Sea $V$ un espacio vectorial  sobre un cuerpo $\mathbb{K}$, $v_1, v_2, v_n$ son vectores de $V$
$k_1, k_2,..., k_n$ son escalares de $\mathbb{K}$.\\
\linebreak
Se dice que un vector $v \in V$ es combinacion lineal de los vectores $v_1, v_2, v_n$ segun los escalares $k_1, k_2, k_n$ si y solo si: 
\begin{equation*}
	v = k_1 v_1+ k_2 v_2+k_n+v_n 
\end{equation*}

\paragraph{Generador de un subespacio vectorial} Sea $V$ un espacio vectorial sobre un cuerpo $\mathbb{K}$,
$v_1, v_2, v_n$ son vectores de $V$, $W$ es el conjunto de \textbf{todas} las combinaciones lineales de $v_1, v_2, v_n$.
Llamaremos a W \textbf{subespacio generado} por $v_1,v_2, v_n$ y lo indicaremos como:
\begin{center}
	$W = <v_1, v_2, v_n>$
\end{center}

y se lee "W es el subespacio generado por $v_1, v_2, v_n$"\\
\linebreak
El subespacio generado por los vectores $v_1, v_2, v_n$, puede ser el mismo $V$, en este caso diremos que el conjunto $S = {v_1, v_2,..., v_n}$ genera el espacio vectorial V o bien, S es un generador de $V$.\\
\linebreak
Sea $V$ un espacio vectorial sobre un cuerpo $\mathbb{K}$, el conjunto $S = {v_1, v_2, v_n}$ es un \textbf{generador} de V si y solo si $v \in V \rightarrow v = k_1v_1 + k_2v_2 + k_nv_n$

\begin{itemize}
	\item Un generador es un conjunto de vectores tales que todo vector del espacio vectorial se puede expresar como una combinacion lineal de ellos
	\item El subespacio generado es el conjunto formado por todas las 
\end{itemize}

\paragraph{Suma de conjuntos}
A y B son 2 conjuntos\\
$A+B = {x=a+b / a \in A y b \in B}$\\

La suma de 2 conjuntos es un nuevo conjunto formado por elementos donde se suma 1 elemento del primer conjunto con 1 elemento del segundo conjunto.

\paragraph{Suma de subespacios vectoriales}
sea V un espacio vectorial sobre un cuerpo $\mathbb{K}$ si $"w_1$ y $w_2$ son subespacios del espacio vectorial V, entonces $W_1 + W_2$ es un subespacio.

\paragraph{Interseccion de subespacios vectoriales}
sea V un espacio vectorial sobre un cuerpo $\mathbb{K}$ si $"w_1$ y $w_2$ son subespacios del espacio vectorial V, entonces $W_1 \cap W_2$ es un subespacio.

\end{document}