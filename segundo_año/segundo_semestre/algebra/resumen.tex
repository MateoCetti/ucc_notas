\title{Resumen de algebra (segundo semestre)}

\author{Mateo P. Cetti}

\documentclass[10pt]{article}

\usepackage{amsmath}
\usepackage{amsfonts}

\begin{document} 
\maketitle

\section{Espacios vectoriales}

cualquier conjunto que posea operaciones de \textbf{suma} y 
\textbf{producto por un escalar}, cumpliendo todas las siguientes \textbf{propiedades:}\\

\linebreak

\paragraph{Suma: }

\begin{enumerate}
    \item \textbf{Asociativa} $(u+w) + v = u (w+v) = (u+v) + w$
    \item \textbf{Conmutativa} $u + v = v + u $
    \item \textbf{Elemento neutro} tal que $u + 0 = u$ 
    \item Para cada vector $u$ existe un elemento opuesto $(-u)$ tal que $u+(-u)=0$.
    
\end{enumerate}

\textbf{Multiplicacion: }

\begin{enumerate}
    \item \textbf{Asociativa} $(k*k')*u=k*(k'*u)=(k*u)*k'  k,k' e \mathbb{K}$
    \item \textbf{Distributiva}
    \begin{itemize}
        \item Respecto a la suma de vectores  $k*(u+v) = k*u+k*v$
        \item Respecto a la suma de escalares  $(k_1+k_2) *u = k_1*u + k_2 *u$   
    
    \end{itemize}
    \item \textbf{Elemento neutro} $k$=1, tal que $1*u = u$
    
\end{enumerate}

Sea $\mathbb{K}$ un cuerpo, y $V$ un conjunto de dos operaciones definimos:

\paragraph{Operacion interna}llamada adición, que asigna a cada par 
$u,v$ de elementos de V, un elemento de V denotado $u+v$

\paragraph{Operacion externa}llamada multiplicación por un escalar que 
asigna a cada par formado por un elemento $ke\mathbb{K}$ y un elemento $veV$, un elemento de V denotado por $kv$

\paragraph{Subespacios vectoriales}

Un subconjunto no vacio W de un espacio vectorial V se denomina subespacio vectorial
de V, si W es en si mismo un espacio vectorial bajo las operaciones de suma y Multiplicacion
por escalar definidas en V. Ademas debe cumplir:

\begin{itemize}
    \item La suma de 2 vectores de W pertenecen a W
    $u,v \mathbb{e} W => u + v \mathbb{e} W$
    \item El producto de un vector de W por un escalar cualquiera K pertenece a W
    $u \mathbb{e} W, k \mathbb{e} \mathbb{K} => k.u  \mathbb{e} W$
    \item El vector nulo pertenece a W 
\end{itemize}

\end{document}