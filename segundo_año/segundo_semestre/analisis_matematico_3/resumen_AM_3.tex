\title{Resumen de analisis matematico III}
\author{
        Mateo P. Cetti \\
        Estudiante - Universdad Catolica de Cordoba\\
        Ing Ambrosio Taravella, 6240, \underline{Cordoba, Argentina}
}
\date{\today}

\documentclass[10pt]{article}

\usepackage{amsmath}
\usepackage{amsfonts}

\begin{document} 
\maketitle

\section{Introduccion [03/08/20]}
En esta meteria vamos a ver funciones como en AM2 solo que esta vez la entrada y salida esta acompañada de \textbf{numeros complejos}
\begin{itemize}
	\item \textbf{Libro:} Variable compleja y sus aplicaciones (Churchil algo)
\end{itemize}

\section{Numeros complejos}
El conjunto $\mathbb C$ de los numeros complejos esta dado como:
\begin{equation*}
	\left\lbrace \mathbb{C}=a+bi   |    a,b    \epsilon \mathbb{R} y i = -1\right\rbrace
\end{equation*}

El simbolo $i$ con la propiedad de que $i^2 = -1$ se denomina unidad \textbf{imaginaria}

Suma y multiplicacion en $\mathbb{C}$:
\begin{itemize}
	\item \textbf{Suma} $(a+bi) + (c+di) = (a+c) + (b+d)i$
	\item \textbf{Multiplicacion} $(a+bi)(c+di) = (ac- bd) + (ad+bc)i$ (menos porque queda $i^2$)
\end{itemize}

Con estas operaciones, el conjunto $\mathbb{C}$ es un \textbf{Conjunto algebraico.}\\

si $z = a+bi$

\begin{itemize}
	\item \textbf{Parte real de z:} a (Notacion: Re[z])
	\item \textbf{Parte imaginaria de z:} b (Notacion: Im[z])
\end{itemize}

\paragraph{Conjugado de un numero complejo (z)}
Dado el numero complejo $z = a+bi$, el numero complejo $x- bi$ se denomina conjugado de $z$ y se denota $\overline{z}$\\

Dado el numero complejo $z = a+bi$, le podemos asignar el par ordenado $(x,y)$, y reciprocamente, dado el par ordenado $(x,y)$, le podemos asignar el numero complejo $z = a+bi$, de modo que existe una \textbf{relacion biunivoca} entre $\mathbb{C}$ y $\mathbb{R}^2$\\

Mirando el plano cartesiano, el eje horizontal se denomina \textbf{eje real} y el eje vertical se denomina \textbf{eje imaginario}. Este plano se denomina \textbf{plano complejo}.\\

\paragraph{Observacion: }Si $a+bi$ = $c+di$ entonces $a=c$ y $b=d$

\paragraph{Modulo de z: } Dado el numero complejo $z = a+bi$, el modulo de $z$ que denotamos $|z|$ esta dado por:

\begin{equation*}
	|z| = \sqrt[2]{a^2+b^2}
\end{equation*}

\begin{center}
	\textbf{Propiedades}
\end{center}
\begin{enumerate}
	\item $|z_1 z_2| = |z_1||z_2|$
	\item $\left|\dfrac{z_1}{z_2}\right| = \dfrac{|z_1|}{|z_2|}$
	\item $|z_1 + z_2| \leq |z_1|+|z_2|$
	\item $z\overline{z} = |z|^2$
	\item $|z_1 - z_2| \geq |z_1|-|z_2|$ 
\end{enumerate}

\begin{center}
\textbf{Una aplicacion de las propiedades de los $\mathbb{C}$}
\end{center}

\begin{equation*}
	\dfrac{1}{a+bi} = \dfrac{a-bi}{(a+bi)(a-bi} = \dfrac{a-bi}{a^2+b^2} = \dfrac{a}{a^2+b^2} + \dfrac{bi}{a^2+b^2}
\end{equation*}
\begin{equation*}
	Re \left[\frac{1}{z}\right] = \dfrac{a}{|z|^2}	
\end{equation*}
\begin{equation*}
	Im \left[\frac{1}{z}\right] = \dfrac{bi}{|z|^2}
\end{equation*}

\paragraph{Forma polar o trigonometrica de un $\mathbb{C}$}

\begin{itemize}
	\item $a = |z|.cos \theta$
	\item $b = |z|.sen \theta$
\end{itemize}

\begin{equation*}
	z = |z|.cos\theta + |z|.i.sen\theta = |z|.(cos\theta+isen\theta) 
\end{equation*}

El Angulo $\theta$ se denomina \textbf{argumento} del numero complejo.\\

\paragraph{Formula de Moivre: }

Si $z = |z|.(cos\theta+isen\theta$ entonces:

\begin{equation*}
	z^n = |z|^n.(cos(n\theta)+isen(n\theta))
\end{equation*}

\paragraph{Funciones complejas}

Sea $f:\mathbb{C}\rightarrow\mathbb{C}$ dada por:

\begin{equation*}
	f(z) = z^2
\end{equation*}

Esto es,

\begin{equation*}
	f(a+bi) = (a+bi)^2 = a^2+b^2+2abi
\end{equation*}

En general:

\begin{center}
	$f(a+bi) = u(a,b)+ iv(a,b)$ Donde $u,v: \mathbb{R}^2 \rightarrow \mathbb{R}$ 
\end{center}

\paragraph{Limite y Continuidad} 
\paragraph{Limite de numeros complejos: }sean $f: \mathbb{C} \rightarrow \mathbb{C}$ y $z_0$ un punto de acumulacion
del dominio de $f$. El numero complejo L es el limite de $f$ en $z_0$ si para todo $\epsilon > 0$ existe
$\delta > 0$, tal que $|f(z)-L| < \epsilon$ cuando $z \in dom(f)$ y $0 < |z-z_0|< \delta$

\paragraph{Continuidad de numeros complejos: }sean $f: \mathbb{C} \rightarrow \mathbb{C}$ y $z_0$ un punto
del dominio de $f$, diremos que $f$ es continua en $z_0$ si para todo $\epsilon > 0$ existe
$\delta > 0$, tal que $|f(z)- f(z_0)| < \epsilon$ cuando $z \in dom(f)$ y $0 < |z-z_0|< \delta$.\\
\begin{equation*}
	\left( \lim_{z \rightarrow z_0} f(z) = f(z_0) \right)
\end{equation*}

\section{Derivada y exponencial compleja}

sean $f: \mathbb{C} \rightarrow \mathbb{C}$ y $z_0 \in dom(f)$. Diremos que $f$ es derivable en $z_0$
si existe $\lim_{z \rightarrow z_0} \dfrac{f(z)-f(z_0)}{z-z_0}$.\\
En este caso se llama a dicho limite \textbf{derivada} de $f$ en $z_0$ y la denotamos $f'(z_0)$.

\paragraph{Ecuaciones de Cauchy - Riemann} (Anda que te las demuestro)

\begin{equation*}
	\dfrac{\partial u}{\partial x} (x_0, y_0) = \dfrac{\partial v}{\partial y} (x_0, y_0)
\end{equation*}

\begin{equation*}
	\dfrac{\partial u}{\partial y} (x_0, y_0) = -\dfrac{\partial v}{\partial x} (x_0, y_0)
\end{equation*}

Sea $f: u+iv$ tal que $u, v: \mathbb{R}^2 \rightarrow \mathbb{R}$ son de clase $C^1$. si $\dfrac{\partial u}{\partial x} (x_0, y_0) = \dfrac{\partial v}{\partial y} (x_0, y_0)$ y $\dfrac{\partial u}{\partial y} (x_0, y_0) = -\dfrac{\partial v}{\partial x} (x_0, y_0)$ en algun conjunto A $\subset dom(u) \cap dom(v)$ entonces $f$ es derivable en el conjunto A.\\
\linebreak

\paragraph{funcion analitica}Sean $f: \mathbb{C} \rightarrow \mathbb{C}$ y $z_0 \in dom(f)$. Diremos que $f$ es \textbf{analitica} en el punto $z_0$ si existe $r > 0$ tal que $B_r(z_0) \subset dom(f)$ y $f$ es derivable en todo punto de $B_r(z_0)$\\
\linebreak

Sean $f: \mathbb{C} \rightarrow \mathbb{C}$ y $z_0 \in dom(f)$. Si $f$ es analitica en $B_r(z_0) - {z_0}$ e dice que $z_0$ es un punto singular o \textbf{Singularidad} de $f$

\paragraph{Exponencial compleja}
Las siguientes condiciones son validas tambien para los numeros complejos:
\begin{enumerate}
	\item $e^0 = 1$
	\item $e^a e^b = e^{a+b} $
	\item $\dfrac{de^x}{dx} = e^x$
\end{enumerate}

\begin{equation*}
	e^{x+yi} = e^{(x+0i)+(0+yi)} = e^{x+0i}e^{0+yi} = e^{x}e^{yi}
\end{equation*}

\paragraph{Formula de Eeuler} (Sin demostracion tambien)
\begin{equation*}
	e^{iy} = cos(y)+isin(y)
\end{equation*}

\paragraph{Propiedades basicas de la exponencial compleja:}

\begin{enumerate}
	\item $\overline{e^z} =e^{\overline{z}}$
	\item $e^z$ es \textbf{analitica} en todo el plano complejo
	\item $e^z \neq 0$ para todo $z \in \mathbb{C}$
	\item $|e^z| = e^{Re(z)}$
\end{enumerate}

\end{document}